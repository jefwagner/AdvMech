\documentclass[12pt]{article}
\usepackage{amsmath}
\usepackage{amssymb}
\usepackage{mathtools}
\usepackage{calc}
\usepackage[margin=1.2in]{geometry}

\pagestyle{empty}
\begin{document}
\begin{center}
Physics 151 Lab, Winter 2015 \hfill 
{\bf Paste this into your lab book}

\vspace{1cm}

{\Large Guildelines for your PHYS151 lab notebook.}
\end{center}

\vspace{1cm}

\begin{enumerate}
\item Read the entire lab and complete the pre-lab {\bf before} class. We all want to finish in three hours; every lab has been finished in this time by students who come prepared! As added incentive, the pre-lab due at the beginning of the lab time, and is worth 10\% of your lab grade.

\item Give a thorough description of the experimental setup, including diagrams and any relevant physical dimensions of the apparatus. Give enough information so that you (or I) could {\bf exactly} reproduce the setup at a later date.

\item When describing calculations made in Excel, state the actual formula you {\bf and} what you typed into Excel. For example,
\begin{equation*}
  s = \sqrt{ \frac{\sum_{i=1}^N(q_i-\bar{q})^2}{N-1}},
  \quad \text{and} \quad \mathtt{STDEV(range)}.
\end{equation*}

\item In the conclusion, list all sources of uncertainty and whether they are random or systematic effects. Comment briefly on what could be attempted to alleviate these effects.

\item Mathematical Modeling: When fitting a curve to a set of data, give a {\bf physical interpretation} of all the best fit parameters, with uncertainties and units.

\item Read the lab before coming to class. This is the most important thing you can do to ensure you finish on time.

\end{enumerate}

\vspace{1cm}

\renewcommand{\arraystretch}{1.2}
\begin{tabular}{|l|l|}
\hline
{\bf Week} & {\bf Lab} \\[1ex] \hline
Jan. 12-16 & 1. Rolling cylinders \\
Jan. 19-23 & 2. Angular momentum \\
Jan. 26-30 & 3. Electrical potential mapping \\
Feb. 2-6 & 4. DC circuits \\
Feb. 9-13 & {\bf NO LAB} mid-term reading period \\
Feb. 16-20 & 5. Charge to mass ratio of the electron \\
Feb. 23-27 & 6. Standing waves \\
Mar. 2-6 & 7. Wavelength of light \\
Mar. 9-13 & 8. Thermal physics \\ \hline
\end{tabular}

\end{document}