\documentclass[10pt]{article}
\usepackage{amsmath}
\usepackage{amssymb}
\usepackage{mathtools}
\usepackage{calc}
\usepackage[margin=1in]{geometry}
\usepackage{pstricks}
\usepackage{multicol}

\newcommand{\dif}{\mathrm{d}}
\renewcommand{\vec}[1]{\mathbf{#1}}

\begin{document}
\thispagestyle{empty}
\makebox{\hspace{1cm}}
\vspace{-0.5in}
\begin{center}
{\Large \bf PHYS425: Homework \#1} \\
Due: Beginning of class, Friday 01/16/2015
\end{center}


\vspace{0.5cm}
\makebox[\textwidth]{Name: \quad \hrulefill}

\vspace{0.5cm}
Please work on your own paper, and staple to this page to turn in.
\begin{multicols}{2}
\begin{enumerate}
\setlength\itemsep{1cm}

\item Show that $\mathcal{L}(q,\dot{q},t)$ and $\mathcal{L}'(q,\dot{q},t)$ give the same Euler-Lagrange equations of motion when
\begin{equation}
\mathcal{L}'(q,\dot{q},t) = \mathcal{L}(q,\dot{q},t) 
+ \frac{\dif}{\dif t} F(q,t)
\end{equation} for an arbitrary function $F(q,t)$.

\item In our derivation in class, we assumed that the potential energy is independent of the velocity. However that is not always the case. Show that the velocity dependent potential energy 
\begin{equation}
U = q \phi - q \vec{A}\cdot\vec{v},
\end{equation}
where $\phi$ and $\vec{A}$ are the (spatially dependent) scalar and vector potentials, which give the electric and magnetic fields with
\begin{align}
\vec{E} & = - \nabla \phi - \frac{\partial}{\partial t}\vec{A}, \\
\vec{B} & = \nabla \times \vec{A},
\end{align}
yields the Lorentz force,
\begin{equation}
\vec{F} = q \vec{E} + q \vec{v} \times \vec{B}.
\end{equation}

\item Find the Euler-Lagrange equations for the compound Atwood machine pictured below. Assuming that all pulleys are massless and lines massless and inextensible, give the acceleration of each of the masses.

\begin{center}
\begin{pspicture}(0,0)(2.2,4)
\pscircle(1.1,3.4){0.5}
\psline(0.6,3.4)(0.6,2.0)
\psline(1.6,3.4)(1.6,2.2)
\pscircle(0.6,1.7){0.3}
\psline(0.3,1.7)(0.3,0.4)
\psline(0.9,1.7)(0.9,1.2)
\psframe(0.0,0.0)(0.6,0.4)
\rput(0.3,0.2){$m_1$}
\psframe(0.6,0.8)(1.2,1.2)
\rput(0.9,1.0){$m_2$}
\pscircle(1.6,1.9){0.3}
\psline(1.3,1.9)(1.3,0.6)
\psline(1.9,1.9)(1.9,1.3)
\psframe(1.0,0.2)(1.6,0.6)
\rput(1.3,0.4){$m_3$}
\psframe(1.6,0.9)(2.2,1.3)
\rput(1.9,1.1){$m_4$}
\end{pspicture}
\end{center}

\item Find the Euler-Lagrange equations for cylinder of radius $R$ and mass $m_1$ that rolls without slipping down a ramp of angle $\alpha$ and mass $m_2$, where the ramp can slide without friction on the level ground, as pictured below. How long does it take for the cylinder a distance $d$ down the ramp.

\begin{center}
\begin{pspicture}(0,0)(4,3)
\psline(0.0,0.26)(4.0,0.26)
\psline(0.1,0.26)(0.3,0.0)
\psline(0.2,0.26)(0.4,0.0)
\psline(0.3,0.26)(0.5,0.0)
\psline(0.4,0.26)(0.6,0.0)
\psline(0.5,0.26)(0.7,0.0)
\psline(0.6,0.26)(0.8,0.0)
\psline(0.7,0.26)(0.9,0.0)
\psline(0.8,0.26)(1.0,0.0)
\psline(0.9,0.26)(1.1,0.0)
\psline(1.0,0.26)(1.2,0.0)
\psline(1.1,0.26)(1.3,0.0)
\psline(1.2,0.26)(1.4,0.0)
\psline(1.3,0.26)(1.5,0.0)
\psline(1.4,0.26)(1.6,0.0)
\psline(1.5,0.26)(1.7,0.0)
\psline(1.6,0.26)(1.8,0.0)
\psline(1.7,0.26)(1.9,0.0)
\psline(1.8,0.26)(2.0,0.0)
\psline(1.9,0.26)(2.1,0.0)
\psline(2.0,0.26)(2.2,0.0)
\psline(2.1,0.26)(2.3,0.0)
\psline(2.2,0.26)(2.4,0.0)
\psline(2.3,0.26)(2.5,0.0)
\psline(2.4,0.26)(2.6,0.0)
\psline(2.5,0.26)(2.7,0.0)
\psline(2.6,0.26)(2.8,0.0)
\psline(2.7,0.26)(2.9,0.0)
\psline(2.8,0.26)(3.0,0.0)
\psline(2.9,0.26)(3.1,0.0)
\psline(3.0,0.26)(3.2,0.0)
\psline(3.1,0.26)(3.3,0.0)
\psline(3.2,0.26)(3.4,0.0)
\psline(3.3,0.26)(3.5,0.0)
\psline(3.4,0.26)(3.6,0.0)
\psline(3.5,0.26)(3.7,0.0)
\psline(3.6,0.26)(3.8,0.0)
\psline(3.7,0.26)(3.9,0.0)
\psline(0.5,0.3)(0.5,2.3)(3.5,0.3)(0.5,0.3)
\pscircle(1.88,2.1333){0.6}
\psline{->}(1.88,2.1333)(2.48,2.1333)
\rput(2.0,2.33333){$R$}
\rput(3.0,0.5){$\alpha$}
\rput(1.5,1.0){$m_2$}
\rput(1.8,1.8){$m_1$}
\end{pspicture}
\end{center}

\item Find the Euler-Lagrange equations for a bead of mass $m$ that slides frictionlessly on vertical hoop or radius $R$, which rotates around a vertical axis with angular velocity $\omega$. Where on the hoop would the bead rest motionless with respect to the hoop? Describe the motion of the bead if it is displaced from this stable point.

\begin{center}
\begin{pspicture}(0,0)(4,5)
\pscircle(2,2.8){2}
\pscircle*(3.6,1.6){0.1}
\psline{->}(2,2.8)(3.6,4)
\psellipticarc{->}(2,0.3)(1,0.3){10}{360}
\psset{linestyle=dashed}
\psline(2,0.3)(2,5)
\rput(2.4,0.3){$\omega$}
\rput(2.8,3.7){$R$}
\rput(3.8,1.4){$m$}
\end{pspicture}
\end{center}


\end{enumerate}
\end{multicols}

\end{document}