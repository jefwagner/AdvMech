\documentclass[10pt]{article}
\usepackage{amsmath}
\usepackage{amssymb}
\usepackage{mathtools}
\usepackage{calc}
\usepackage[margin=0.8in]{geometry}

\begin{document}
\thispagestyle{empty}
\mbox{\hspace{1in}}
\vspace{-0.3in}
\begin{center}
{\Large \bf PHYS425: Advanced Mechanics and Computational Physics}
\end{center}

\vspace{0.3cm}

\begin{tabular}{ll}
{\bf Instructor:} & Jef Wagner \\
{\bf Classroom:} & Youngchild Hall 041 \\
{\bf Time:} & 11:10-12:20 MWF \\
{\bf Book:} & Fowles and Cassiday, {\it Analytical Mechanics} $6^{\text{th}}$ or $7^{\text{th}}$ edition \\
{\bf Office Hours:} & 2:00pm-4:00pm Monday, Tuesday, and Wednesday
\end{tabular}

\vspace{-0.1cm}
\subsubsection*{Course Goals:}
As is hopefully obvious from the course title, this course is a combination of two separate topics: advanced mechanics and computational physics, therefore there are two separate goals. 

{\bf Advanced mechanics:} you will further build a foundation in physics/mechanics. Specifically we will cover the Lagrangian/Hamiltonian formulation of mechanics, mechanics in non-inertial reference frames, small oscillations and normal modes, and simple perturbation theory. 

{\bf Computational physics:} you will become comfortable working with and visualizing computational solutions. Specifically you will learn to use the scientific programming package SciPy in the Python programming language, and you will learn to use Mathematica to visualize the solutions.

\vspace{-0.1cm}
\subsubsection*{Grading:}
\begin{minipage}{0.67\textwidth}
The final grade will be combined from the portions to the right. At the end of the term, I will try to find a natural division to assign grades of {\bf A}, {\bf B}, {\bf C}, {\bf D}, and {\bf F} (I will not assign any + for - grades). However, I do guarantee that final grades between 100\%-90\% will earn an {\bf A}, 89.9\%-80\% at least a {\bf B}, 79.9\%-70\% at least a {\bf C}, and 69.9\%-60\% will earn at least a {D}.
\end{minipage}
\begin{minipage}{0.03\textwidth}
\mbox{\hspace{0.1in}}
\end{minipage}
\begin{minipage}{0.3\textwidth}
\begin{tabular}{ll}
In-class work: & 20\% \\
Homework: & 20\% \\
Computational Projects: & 20\% \\
Midterm: & 20\% \\
Final: & 20\%
\end{tabular}
\end{minipage}

\vspace{-0.1cm}
\subsubsection*{In-class work:}
I will not be taking attendance, however part of your grade comes from in-class work. I will drop the two lowest grades from the in class work, so that gives you two unexcused absences.

\vspace{-0.1cm}
\subsubsection*{Homework:}
Homework will be graded on completion, and late homework will count at 80\%. I encourage you to work together, but everyone will be responsible for turning in their own work. I will post solutions the day the homeworks are due.

\vspace{-0.1cm}
\subsubsection*{Computational Projects:}
You will work in groups of two in order to complete four computational projects during the term. The projects will be approximately spaced every two weeks with a break for the midterm. All projects will cover both a computational and visualization portion, and will be graded over both correctness and quality of the visualization.

\vspace{-0.1cm}
\subsubsection*{Midterm:}
The date of the midterm is yet to be determined. The midterm will consist of an in-class and take home portion. You may use your notes, book, and even other resources (google/wikipedia/stackoverflow). However, you may NOT work together, and you will be expected to write all your code yourself (no cutting and pasting from the internet).

\vspace{-0.1cm}
\subsubsection*{Final:}
The final will be Wednesday, March 18th from 8:00am-10:30am in Youngchild 041 (same as our meeting room). The final will be cumulative, this means that I reserve the right to ask questions that cover the same subjects covered in the midterm.

\vspace{-0.1cm}
\subsubsection*{Accommodations:} All students are entitled to a education in the least restrictive environment possible, which includes accommodations for those students with disabilities. All qualified students should contact me informing me of their specific needs before the second week of classes.

\end{document}